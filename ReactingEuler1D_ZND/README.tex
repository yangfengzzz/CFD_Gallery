\documentclass[preprint,12pt]{elsarticle}

\usepackage[a4paper,margin=0.5in]{geometry}
\usepackage{bm}

\usepackage{amssymb}
\usepackage{amsmath}
\newtheorem{remark}{Remark}
\newtheorem{example}{Example}
\newtheorem{theorem}{Theorem}
\usepackage{subfigure}


\begin{document}
In numerical study of detonation phenomenon with reactive Euler equations,
numerical solution of 1D ZND model is often used as the initial condition.
Below we briefly introduce the ZND model as well as the derivation of the
solution.

In ZND model, the detonation process is considered as the convection
f the mixture together with the transformation of the mixture from
the reactant to the product. Hence, the governing equations consist of
the conservation laws and the balance laws. In this paper, for the
reaction process, we assume that there are only reactant and product
species, and that the reaction is irreversible. Then in 1D case, the
reactive Euler equations are given by
\begin{subequations}\label{1D RE PDE}
  \begin{align}
    \displaystyle\frac{\partial}{\partial t}\rho + \frac{\partial}{\partial x}(\rho u) &= 0,\label{1D RE PDE:a}\\
    \displaystyle\frac{\partial}{\partial t}(\rho u) + \frac{\partial}{\partial x}(\rho u^2 + P) &= 0,\label{1D RE PDE:b}\\
    \displaystyle\frac{\partial}{\partial t}(E) + \frac{\partial}{\partial
    x}(u(E + P)) &= 0,\label{1D RE PDE:c}\\
    \displaystyle\frac{\partial}{\partial t}(\rho Y) + \frac{\partial}{\partial x}(\rho u Y) &= \omega,\label{1D RE PDE:d}
  \end{align}
\end{subequations}

In the following, we assume that a detonation wave is propagating with a
constant velocity $s$ along the $x$-direction of a tube, and that the flow is
steady with respect to a coordinate system moving with the wave. From now on, a
variable with subscript $u$ denotes its unburnt status, while with subscript
$b$ for its complete burnt status. For example, $\rho_u$ denotes the density of
the unburnt gas at the downstream side of the tube ($x=+\infty$), while
$\rho_b$ denotes the density of the completely burnt gas at the upstream side
of the tube ($x = -\infty$). With the given unburnt state of the solutions and
parameters, the task now is to determine the velocity $s$ and the burnt state
of the solutions, and this can be done by ZND theory as follows.

First of all, a traveling wave coordinate $\xi = x - st$ is introduced, and the
above equations (\ref{1D RE PDE}) can be transferred to the following ordinary
differential equations

\begin{subequations}\label{1D RE ODE}
  \begin{align}
    \displaystyle -s\frac{d}{d\xi}(\rho) + \frac{d}{d\xi}(\rho u) &= 0,\label{1D RE ODE:a}\\
    \displaystyle -s\frac{d}{d\xi}(\rho u) + \frac{d}{d\xi}(\rho u^2 + P) &= 0,\label{1D RE ODE:b}\\
    \displaystyle -s\frac{d}{d\xi}(E) + \frac{d}{d\xi}(u(E + P)) &= 0,\label{1D RE ODE:c}\\
    \displaystyle -s\frac{d}{d\xi}(\rho Y) + \frac{d}{d\xi}(\rho u Y) &= \omega.\label{1D RE ODE:d}
  \end{align}
\end{subequations}

By introducing the specific volume $V = 1/\rho$, it can be derived
from (\ref{1D RE ODE:a}) and (\ref{1D RE ODE:b}) the so-called
Rayleigh line
\begin{equation}\label{Rayleigh Line}
  P(V) = -m^2(V - V_u) + P_u,
\end{equation}
where $m = \rho_u(s - u_u)$ is the mass flux. Then the Hugoniot curve
can be derived from (\ref{1D RE ODE}), by using the the equation of
state, and the quantity enthalpy $h = E - \rho u^2/2 + PV$, as
follows
\begin{equation}
  P(V, Y) = \left (2Q(1 - Y) - P_uV + \displaystyle\frac{\gamma + 1}{\gamma - 1}P_uV_u\right ) \Big/ \left (\displaystyle\frac{\gamma + 1}{\gamma - 1}V-V_u\right ).
\end{equation}

Fig. \ref{Hugo Rayleigh line} shows a classical process of a strong ZND
detonation. Briefly, the reactant gas is compressed by the leading shock, and
the reactant state changes from $(V_u, P_u)$ (solid circle point) to the Neumann
point (solid square point) along the Hugoniot curve (dashed one) in Fig.
\ref{Hugo Rayleigh line}. Then the reaction process starts, and the reactant
state changes from the Neumann point (solid square point) to the point $(V_b,
P_b)$ (solid triangle point) along the upper dashed Rayleigh line. It is noted
that the minimum velocity for a detonation wave is the speed of Chapman-Jouguet
(CJ) detonation. Similarly, the CJ detonation corresponds to two process showed
in Fig. \ref{Hugo Rayleigh line}, i.e., the leading shock changes the reactant
state from the point $(V_u, P_u)$ (solid circle point) to the Neumann point
(star point) along the Hugoniot curve (dashed one), then the chemical reaction
changes the reactant state from the Neumann point (star point) to the final
state $(V^{CJ}_b, P^{CJ}_b)$ (solid diamond point).

Hence, with given $\rho_u$, $u_u$, $P_u$, as well
as the parameter $\gamma$, $Q$, and $f$, the detonation velocity $s$ as well as
the mass flux $m$ are obtained as follows. First of all, the mass flux of CJ
detonation, $m_{CJ}$, is defined by

\begin{equation}
  m_{CJ}^2 = \gamma\frac{P_u}{V_u} + (\gamma^2 - 1)\frac{Q}{V_u^2}\left (1 +
    \sqrt{1 + \frac{2\gamma P_uV_u}{(\gamma^2 - 1)Q}}\right ).
\end{equation}
Then the velocity of CJ detonation wave is given by
\begin{equation}
  s_{CJ} = \frac{\rho_u u_u+ m_{CJ}}{\rho_u}.
\end{equation}
To define a strong detonation, an over-driven factor $f$ is introduced to build
the relation between the strong detonation speed $s$ and the CJ detonation speed
$s_{CJ}$ as $s^2=fs_{CJ}^2$. From (\ref{1D RE PDE:a}), (\ref{1D RE PDE:d}), and (\ref{1D RE ODE:a}), the
following ODE
equation can be derived for the distribution of the mass fraction $Y$ in the
domain,
\begin{equation}\label{ODE4Y}
  \left \{ \begin{array}{l}
    \displaystyle\frac{d}{d\xi}Y = - \frac{\omega}{m},\quad \forall \xi<0\\
    \\
    Y(0) = 1.
  \end{array}\right .
\end{equation}
By product rule of the calculus, (\ref{1D RE ODE:d}) gives 
\begin{equation}
  -sY\frac{d\rho}{d\xi} - s\rho\frac{dY}{d\xi} + Y\frac{d\rho u}{d \xi} + \rho
  u\frac{dY}{d\xi} = w.
\end{equation}
Then replacing the first term on the left side of the above equation by (\ref{1D RE ODE:a}), it follows that
\begin{equation}
  -Y\frac{d\rho u}{d\xi} - s\rho\frac{dY}{d\xi} + Y\frac{d\rho u}{d \xi} + \rho
  u\frac{dY}{d\xi} = w.
\end{equation}
After the simplification, we have the following ODE for $Y$
\begin{equation}
    \frac{dY}{d\xi} = \frac{w}{\rho(u - s)} = -\frac{w}{m}.
\end{equation}
with the initial condition $Y(0) = 1$.

Finally, with the mass fraction $Y$, all other quantities are given by
\begin{equation}\label{derived Quantities}
  \begin{array}{l}
    P(Y) = \displaystyle\frac{m^2v_u + P_u}{\gamma + 1} + \frac{1}{\gamma + 1}\beta(Y),\\
    \\
    V(Y) = \displaystyle\frac{\gamma(m^2V_u + P_u)}{m^2(\gamma + 1)} - \frac{1}{m^2(\gamma + 1)}\beta(Y),\\
    \\
    u(Y) = s - mV(Y),
  \end{array}
\end{equation}
where $\beta(Y)$ is given by
\begin{equation}
  \beta(Y) = \sqrt{(m^2V_u - \gamma P_u)^2 - 2(\gamma^2 - 1)m^2Q(1 - Y)}.
\end{equation}
It is noted that the final state of the solutions can be read from the above
functions with $Y=0$.

\end{document}
\endinput